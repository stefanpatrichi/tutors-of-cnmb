\documentclass{scrartcl}
\usepackage[sexy]{evan}
\usepackage[romanian]{babel}
\usepackage{algpseudocode}
\author{Test Tutors of CNMB Informatică}
\date{\today}

\begin{document}
\section*{Problema \texttt{autostrada} (10p)}
Ciorogârla se află la kilometrul 60 pe autostrada A2. Se știe că noi suntem la kilometrul $k$.
\section*{Cerință}
La ce distanță de Ciorogârla suntem?
\section*{Date de intrare}
Se citește $k$.
\section*{Date de ieșire}
Se va afișa distanța până la Ciorogârla.
\section*{Restricții și precizări}
\begin{itemize}
    \item $0 \leq k \leq 10^9$
\end{itemize}

{
\parindent0pt
    
\section*{Exemplul 1}
\texttt{Intrare}:
    \begin{lstlisting}
34
    \end{lstlisting}

\texttt{Ieșire}:
    \begin{lstlisting}
22
    \end{lstlisting}

}

\pagebreak

\section*{Problema \texttt{divide} (20p)}
\section*{Cerință}
Dându-se două numere întregi, $n$ și $k$, să se afișeze câtul împărțirii $\frac{n}{k}$, în cazul în care $k$ divide $n$.
\section*{Date de intrare}
Se citesc $n$ și $k$ întregi.
\section*{Date de ieșire}
Se va afișa câtul împărțirii $\frac{n}{k}$, în cazul în care $k$ divide $n$, altfel -1.
\section*{Restricții și precizări}
\begin{itemize}
    \item $-10^9 \leq n, k \leq 10^9$
\end{itemize}

{
\parindent0pt
    
\section*{Exemplul 1}
\texttt{Intrare}:
    \begin{lstlisting}
45 9
    \end{lstlisting}

\texttt{Ieșire}:
    \begin{lstlisting}
5
    \end{lstlisting}

}

\pagebreak

\section*{Problema \texttt{inecuatie} (25p)}
Gigel știe că o inecuație are mulțimea soluțiilor $\mathcal{S} = \mathcal{S}_1 \cup \mathcal{S}_2$.
În plus, $\mathcal{S}_1 = [a, b)$ și $\mathcal{S}_2 = (c, d]$.
\section*{Cerință}
Gigel vrea să afle dacă o anumită valoare este soluție a inecuației.
\section*{Date de intrare}
Se citesc $a$, $b$, $c$, $d$ și $x$. 
\section*{Date de ieșire}
Să se afișeze \texttt{sol} dacă $x \in \mathcal{S}$ și \texttt{nu e sol} dacă $x$
nu este soluție a inecuației.
\section*{Restricții și precizări}
\begin{itemize}
    \item $-10^9 \leq a < b < c < d \leq 10^9$
    \item $-10^9 \leq x \leq 10^9$
\end{itemize}

{
\parindent0pt
    
\section*{Exemplul 1}
\texttt{Intrare}:
    \begin{lstlisting}
2 5 10 13
5
    \end{lstlisting}

\texttt{Ieșire}:
    \begin{lstlisting}
nu e sol
    \end{lstlisting}
}

\pagebreak

\section*{Problema \texttt{progress} (25p)}
Se dau trei numere pozitive, $x$, $y$ și $z$.
\section*{Cerință}
Numerele, în această ordine, se află în progresie artimetică (cerința 1)? Dar geometrică (cerința 2)?
\section*{Date de intrare}
Se citesc $c$, $x$, $y$ și $z$.
\section*{Date de ieșire}
Dacă $c = 1$, în cazul în care $x$, $y$ și $z$ sunt în progresie artimetică, afișați rația acesteia, altfel afișați \texttt{X}.

\noindent Dacă $c = 2$, în cazul în care $x$, $y$ și $z$ sunt în progresie geometrică, afișați rația acesteia, altfel afișați \texttt{X}.
\section*{Restricții și precizări}
\begin{itemize}
    \item $c = 1$ sau $c = 2$
    \item $0 \leq x,y,z \leq 10^9$
\end{itemize}

{
\parindent0pt
    
\section*{Exemplul 1}
\texttt{Intrare}:
    \begin{lstlisting}
1
2 4 6
    \end{lstlisting}

\texttt{Ieșire}:
    \begin{lstlisting}
2
    \end{lstlisting}

\section*{Exemplul 2}
\texttt{Intrare}:
    \begin{lstlisting}
2
3 9 29
    \end{lstlisting}

\texttt{Ieșire}:
    \begin{lstlisting}
X
    \end{lstlisting}

}

\section*{Problema \texttt{ident} (20p)}
Se dă algoritmul:

\begin{algorithmic}[1]
    \State \texttt{\textbf{citește} a} \texttt{(a} $\geq 1000$, \texttt{a} $\in \mathbb{N}$\texttt{)}
    \State \texttt{aux $\gets$ a \% 100};
    \State \texttt{a $\gets$ a / 1000};
    \State \texttt{a $\gets$ a + aux * 100};
    \State \texttt{\textbf{scrie} a};
\end{algorithmic}

\section*{Cerință}
\begin{enumerate}[a)]
    \item (10p) Execută algoritmul pentru \texttt{a = 32958} și \texttt{a = 4692}.
    \item (10p) Care este efectul algoritmului?
\end{enumerate}    



\end{document}