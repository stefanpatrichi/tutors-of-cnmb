\documentclass{scrartcl}
\usepackage[sexy]{evan}
\usepackage[romanian]{babel}
\author{Test Tutors of CNMB Informatică}
\date{17 februarie 2024}

\begin{document}
\section*{Problema \texttt{pass}}
În calitate de informaticieni pricepuți, ați fost selectați de conducerea CNMB pentru a-i ajuta la digitalizarea catalogului școlar,
în special în determinarea celor care au trecut clasa și a celor care au rămas corigenți.
\section*{Cerință}
Dându-se o medie, să se afișeze dacă elevul care a obținut-o a trecut clasa sau este corigent.
\section*{Date de intrare}
Se va citi de la tastatură media $m$.
\section*{Date de ieșire}
Se va afișa \texttt{trece} dacă elevul care a obținut media $m$ a trecut clasa și \texttt{corigent} dacă a rămas corigent.
(exact așa cum este în enunț, cu litere mici!)
\section*{Restricții și precizări}
\begin{itemize}
    \item $1 \leq m \leq 10$, $m \in \mathbb{N}$
    \item nota minimă de trecere este 5
\end{itemize}

{
\parindent0pt
    
\section*{Exemplul 1}
\texttt{stdin}:
    \begin{lstlisting}
3
    \end{lstlisting}

\texttt{stdout}:
    \begin{lstlisting}
corigent
    \end{lstlisting}

\section*{Exemplul 2}
\texttt{stdin}:
    \begin{lstlisting}
8
    \end{lstlisting}

\texttt{stdout}:
    \begin{lstlisting}
trece
    \end{lstlisting}
}
\end{document}