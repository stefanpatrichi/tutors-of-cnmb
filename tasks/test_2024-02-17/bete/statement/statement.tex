\documentclass{scrartcl}
\usepackage[sexy]{evan}
\usepackage[romanian]{babel}
\author{Test Tutors of CNMB Informatică}
\date{17 februarie 2024}

\begin{document}
\section*{Problema \texttt{bete}}
Trei bețe au lungimile de $a$ cm, $b$ cm, respectiv $c$ cm. Tăiem din bățul cel mai lung o bucată egală cu bățul cel mai scurt. 
\section*{Cerință}
Calculați diferența (în modul) de lungime dintre bucata rămasă și bățul de lungime medie.
\section*{Date de intrare}
Se citesc dimensiunile celor trei bețe: $a$, $b$ și $c$.
\section*{Date de ieșire}
Se afișează un număr reprezentând răspunsul la cerință.
\section*{Restricții și precizări}
\begin{itemize}
    \item $1 \leq a, b, c \leq 10^7$
    \item $a, b, c$ distincte două câte două
\end{itemize}

{
\parindent0pt
    
\section*{Exemplul 1}
\texttt{stdin}:
    \begin{lstlisting}
180 100 240
    \end{lstlisting}

\texttt{stdout}:
    \begin{lstlisting}
40 
    \end{lstlisting}

\subsection*{Explicație}
Maximul este 240, minimul este 100, iar răspunsul este 40 deoarece $180-(240-100) = 40$.
\includegraphics[scale=0.6]{../images/bete.png}

\section*{Exemplul 2}
\texttt{stdin}:
    \begin{lstlisting}
50 150 90
    \end{lstlisting}

\texttt{stdout}:
    \begin{lstlisting}
10
    \end{lstlisting}
\subsection*{Explicație}
$(150-50)-90 = 10$.

}
\end{document}