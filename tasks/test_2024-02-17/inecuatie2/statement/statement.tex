\documentclass{scrartcl}
\usepackage[sexy]{evan}
\usepackage[romanian]{babel}
\author{Test Tutors of CNMB Informatică}
\date{17 februarie 2024}

\begin{document}
\section*{Problema \texttt{inecuatie2}}
Gigel își amintește de problema \texttt{inecuatie} de la testul inițial. Era vorba de o
inecuație cu mulțimea soluțiilor $\mathcal{S} = \mathcal{S}_1 \cup \mathcal{S}_2$, iar
$\mathcal{S}_1 = [a, b)$ și $\mathcal{S}_2 = (c, d]$.
\section*{Cerință}
Acum, Gigel vrea să știe dacă o anumită valoare face parte din $\mathcal{S}_1$, $\mathcal{S}_2$ sau niciuna.
\section*{Date de intrare}
Se citesc $a$, $b$, $c$, $d$ și $x$. 
\section*{Date de ieșire}
Să se afișeze \texttt{0} dacă $x \notin \mathcal{S}$, \texttt{1} dacă $x \in \mathcal{S}_1$ și \texttt{2} dacă $x \in \mathcal{S}_2$.

\section*{Restricții și precizări}
\begin{itemize}
    \item $-10^9 \leq a < b < c < d \leq 10^9$
    \item $-10^9 \leq x \leq 10^9$

\end{itemize}

{
\parindent0pt
    
\section*{Exemplul 1}
\texttt{stdin}:
    \begin{lstlisting}
2 5 10 13 
4
    \end{lstlisting}

\texttt{stdout}:
    \begin{lstlisting}
1
    \end{lstlisting}

}
\end{document}